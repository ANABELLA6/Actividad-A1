Párrafo 1: Contexto y Desafío de la Gestión de Requerimientos
En el mundo actual de desarrollo de software, la gestión efectiva de los requerimientos es esencial para el éxito de cualquier proyecto. Los requerimientos actúan como el fundamento sobre el cual se construye todo el proceso de desarrollo, y su correcta identificación, documentación y seguimiento son cruciales para evitar desviaciones, retrasos y costos innecesarios. Sin embargo, a menudo nos encontramos con desafíos en la captura precisa de requerimientos, la comunicación efectiva entre los equipos y la garantía de que los cambios sean gestionados de manera controlada. Es en este contexto que surge la necesidad.

Párrafo 5: Alcance y Resultados Esperados El alcance del proyecto abarcará desde la definición de requerimientos hasta la implementación y prueba del software resultante. Se espera que el programa sea capaz de gestionar múltiples tipos de requerimientos, permitir la trazabilidad de cambios, facilitar la colaboración entre equipos y generar informes analíticos útiles. Los resultados esperados incluyen un software funcional, documentación detallada, capacitación para los usuarios y una transición sin problemas hacia la adopción del nuevo sistema de gestión de requerimientos.e desarrollar un programa de software que agilice y mejore significativamente la gestión de procesos de requerimientos.

